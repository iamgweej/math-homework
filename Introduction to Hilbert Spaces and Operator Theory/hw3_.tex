% !TEX TS-program = pdflatex
% !TEX encoding = UTF-8 Unicode

% This is a simple template for a LaTeX document using the "article" class.
% See "book", "report", "letter" for other types of document.

\documentclass[11pt]{article} % use larger type; default would be 10pt

\usepackage[utf8]{inputenc} % set input encoding (not needed with XeLaTeX)

%%% Examples of Article customizations
% These packages are optional, depending whether you want the features they provide.
% See the LaTeX Companion or other references for full information.

%%% PAGE DIMENSIONS
\usepackage{geometry} % to change the page dimensions
\geometry{a4paper} % or letterpaper (US) or a5paper or....
% \geometry{margin=2in} % for example, change the margins to 2 inches all round
% \geometry{landscape} % set up the page for landscape
%   read geometry.pdf for detailed page layout information

\usepackage{graphicx} % support the \includegraphics command and options

% \usepackage[parfill]{parskip} % Activate to begin paragraphs with an empty line rather than an indent

%%% PACKAGES
\usepackage{booktabs} % for much better looking tables
\usepackage{array} % for better arrays (eg matrices) in maths
\usepackage{paralist} % very flexible & customisable lists (eg. enumerate/itemize, etc.)
\usepackage{verbatim} % adds environment for commenting out blocks of text & for better verbatim
\usepackage{subfig} % make it possible to include more than one captioned figure/table in a single float
\usepackage{amsfonts}
\usepackage{amsmath}
\usepackage{amsthm}
\usepackage[lastexercise]{exercise}
% These packages are all incorporated in the memoir class to one degree or another...

%%% Commands and Theorems
\newcommand{\norm}[1]{\left\lVert#1\right\rVert}
\newcommand{\abs}[1]{\left|#1\right|}
\newcommand{\inpr}[2]{\left<#1, #2\right>}
\newtheorem{lemma}{Lemma}

%%% HEADERS & FOOTERS
\usepackage{fancyhdr} % This should be set AFTER setting up the page geometry
\pagestyle{fancy} % options: empty , plain , fancy
\renewcommand{\headrulewidth}{0pt} % customise the layout...
\lhead{}\chead{}\rhead{}
\lfoot{}\cfoot{\thepage}\rfoot{}

%%% SECTION TITLE APPEARANCE
\usepackage{sectsty}
\allsectionsfont{\sffamily\mdseries\upshape} % (See the fntguide.pdf for font help)
% (This matches ConTeXt defaults)

%%% ToC (table of contents) APPEARANCE
\usepackage[nottoc,notlof,notlot]{tocbibind} % Put the bibliography in the ToC
\usepackage[titles,subfigure]{tocloft} % Alter the style of the Table of Contents
\renewcommand{\cftsecfont}{\rmfamily\mdseries\upshape}
\renewcommand{\cftsecpagefont}{\rmfamily\mdseries\upshape} % No bold!

%%% END Article customizations

%%% The "real" document content comes below...

\title{Homework 3}
\author{Alon Ben-Tsur}
%\date{} % Activate to display a given date or no date (if empty),
         % otherwise the current date is printed 

\begin{document}
\maketitle

\begin{Exercise}
\begin{enumerate}
\item Show that $\left(\mathbb{R}^2, \norm{\cdot}_1 \right)$ and $\left(\mathbb{R}^2, \norm{\cdot}_\infty \right)$ are isometric.
\item Prove that an isometry maps extreme points to extreme points.
\item Show that $\left(\mathbb{R}^3, \norm{\cdot}_1 \right)$ and $\left(\mathbb{R}^2, \norm{\cdot}_\infty \right)$ are not isometric.
\end{enumerate}
\end{Exercise}

\begin{Answer}
\begin{enumerate}
\begin{item}
Consider the map $T: \left(x, y\right) \mapsto \frac{1}{2}\left(x + y, x - y\right)$ from $\left(\mathbb{R}^2, \norm{\cdot}_1 \right)$ to $\left(\mathbb{R}^2, \norm{\cdot}_\infty \right)$. It's clearly linear and onto.

Now, suppose $\abs{x} \geq \abs{y}$. If $x \geq 0$ we have that:
\[ \norm{Tv}_\infty = \frac{1}{2}\abs{x + y} + \frac{1}{2}\abs{x-y} = \frac{1}{2}\left(x + y\right) + \frac{1}{2}\left(x - y\right) = x = \norm{v}_1 \]

When $x < 0$, we have that:
\[ \norm{Tv}_\infty = \frac{1}{2}\abs{x + y} + \frac{1}{2}\abs{x-y} = -\frac{1}{2}\left(x + y\right) + -\frac{1}{2}\left(x - y\right) = -x = \norm{v}_1 \]

When $\abs{y} \geq \abs{x}$ we can reduce to the first case by composing with the isometry $\left(x,y\right) \mapsto \left(y,x\right)$.

We conclude that $T$ is an isometry and so $\left(\mathbb{R}^3, \norm{\cdot}_1 \right)$ and $\left(\mathbb{R}^3, \norm{\cdot}_1\infty \right)$ are isometric.
\end{item}
\begin{item}
Take an isometry $T: X\to Y$ and an extreme point $x\in X$. First of all $\norm{Tx} = \norm{x} \leq 1$, so $Tx$ is in the unit ball in $Y$. Take $y_1, y_2 \in B\left(Y\right)$ and $\lambda \in \left(0, 1\right)$ with $\lambda y_1 + \left(1 - \lambda\right)y_2 = Tx$. Then $\norm{T^{-1}y_i} = \norm{y_i} \leq 1$, so $T^{-1}y_1, T^{-1}y_2 \in B\left(X\right)$. Since $T$ is linear, we have that $\lambda T^{-1}y_1 + \left(1-\lambda\right)T^{-1}y_2 = x$, so $T^{-1}y_1 = T^{-1}y_2$ and finally $y_1 = y_2$. We conclude that $Tx$ is extreme.
\end{item}

\begin{item}
There are six extreme points in $\left(\mathbb{R}^3, \norm{\cdot}_1 \right)$, namely $\pm e_i$ for $i=1,2,3$. In the meantime, there are eight extreme points in $\left(\mathbb{R}^3, \norm{\cdot}_\infty \right)$, those being $\pm e_i \pm e_j$ for $i,j = 1,2,3, i\neq j$. Therefore, there cant be a one-to-one mapping between $\left(\mathbb{R}^3, \norm{\cdot}_1 \right)$ and $\left(\mathbb{R}^3, \norm{\cdot}_\infty \right)$ mapping extreme points to extreme points, and in particual no isometries between those spaces can exist.
\end{item}
\end{enumerate}
\end{Answer}

\begin{Exercise}
Prove that a normed space is complete if and only if for every sequence of vectors $x_n$ such that $\sum_n \norm{x_n} < \infty$ it follows that $\sum_n x_n < \infty$.
\end{Exercise}

\begin{Answer}
Assume $X$ is a complete normed space and take a sequence of vectors $x_n$ such that the series $\sum_n \norm{x_n}$ converges. Then by Cauchy's criteria, we know that for every $\varepsilon > 0$ there exists some $N > 0$ such that for all $m > n > N$ we have that $\sum_{k=n}^m \norm{x_k} < \varepsilon$. From the triangle ineqaulity, we have that $\norm{\sum_{k=n}^m x_k} < \varepsilon$, so the sequence $S_n = \sum_{k = 1}^n x_k$ is a Cauchy sequence in $X$. From completeness, the sequence $\sum_n x_n$ converges.

Assume that for every sequence of vectors $x_n$ with $\sum_n \norm{x_n} < \infty$ we have that $\sum_n x_n < \infty$ in $X$. Take a Cauchy sequence $y_n$ in $X$. For each $k > 0$, there is an $N_k > 0$ such that for all $n, m > N_k$ we have that $\norm{y_n - y_m} < 2^{-k}$. Set $n_1 = N_1$ and $n_{k+1} = \max\left\{n_k, N_k\right\} + 1$. Then by definition, the it holds that $\norm{y_{n_{k+1}} - y_{n_k}} < 2^{-k}$ for al $k >0$. Define $x_k = y_{n_{k+1}} - y_{n_k}$. Then we have that $\sum_k \norm{x_k} \leq \sum_k 2^{-k} = 1$, and from our assumption the series $\sum_k x_k$ converges to some $x\in X$. We have that $\sum_{k=1}^n x_k = y_{n_{k+1}} - y_{n_1}$, therefore $y_{n_k} \to x + y_{n_1} = y$ in $X$. Now, for some $\varepsilon > 0$, take $N > 0$ such that for all $n,m > N$ we have that $\norm{y_n -y_m} < \varepsilon$. Then in particular, for large enough $k$, we have that $\norm{y_n -y_{n_k}} < \varepsilon$ and letting $k \to \infty$ we have that $\norm{y_n -y} \leq \varepsilon$. Therefore, $y_n \to y$, so $X$ is complete. 
\end{Answer}

\begin{Exercise}
\begin{enumerate}
\begin{item}
Prove that in an inner product space (over $\mathbb{C}$), for all vector $x,y$ we have
\[\inpr{x}{y} = \frac{1}{4} = \left(\norm{x + y}^2 - \norm{x - y}^2 + i \norm{x + iy}^2 - i \norm{x - iy}^2 \right) \]
\end{item}
\item Prove that $\left(C\left[0,1\right], \norm{\cdot}\right)$ is not an inner product space.
\end{enumerate}
\end{Exercise}

\begin{Answer}
\begin{enumerate}
\begin{item}
Take $w,z$ vectors. Then we have that:
\[ \norm{w + z}^2 =  \inpr{w+z}{w+z} = \norm{w}^2 + \norm{z}^2 + 2\Re\inpr{w}{z} \]
So the expression in the exercise reduces to:
\[ 2\Re\inpr{x}{y} - 2\Re\inpr{x}{-y} + 2i\Re\inpr{x}{iy} -2i\Re\inpr{x}{-iy}\]
Noticing that $\inpr{x}{-y} = -\inpr{x}{y}$ and that $\inpr{x}{iy} = -i\inpr{x}{y}$ we can substitute:
\[ 2\Re\inpr{x}{y} + 2\Re\inpr{x}{y} + 2i\Im\inpr{x}{y} +2i\Im\inpr{x}{y} \]
That is, $4\inpr{x}{y}$.
\end{item}
\begin{item}
Define  the following functions in $C\left[0,1\right]$:

\[
f\left(x\right) = 
\begin{cases}
16x, & 0 \leq x < \frac{1}{4} \\
16\left(\frac{1}{2} - x\right), & \frac{1}{4} \leq x < \frac{1}{2} \\
0, & \frac{1}{2} \leq x \leq 1
\end{cases}
\]

\[
g\left(x\right) = 
\begin{cases}
0, & 0 \leq x < \frac{1}{2} \\
16\left(x - \frac{3}{4}\right), & \frac{1}{2} \leq x < \frac{3}{4} \\
16\left(1 - x\right), & \frac{3}{4} \leq x \leq 1
\end{cases}
\]

These are two triangles of height $4$, the first based on $\left[0,\frac{1}{2}\right]$, the second based on $\left[\frac{1}{2}, 1\right]$.

Assume by contradiction that the Parallelogram law defines an inner product on $\left(C\left[0,1\right], \norm{\cdot}_1\right)$. Then we would have that:
\[ \inpr{f}{g} = \frac{1}{4}\left(\norm{f+g}_1^2 - \norm{f-g}_1^2 + i\norm{f + ig}_1^2 - i\norm{f - ig}_1^2\right) = 0 \]

And:
\[ \inpr{f+g}{g} = \frac{1}{4}\left(\norm{f + 2g}_1^2 - \norm{f}_1^2 + i\norm{f + \left(1 + i\right)g}_1^2- i\norm{f + \left(1 - i\right)g}_1^2\right) = 2 \]

So we have that $2 = \inpr{f+g}{g} = \inpr{f}{g} + \inpr{g}{g} = 0 + 1 = 1$, which is a contradiction.

\end{item}
\end{enumerate}
\end{Answer}

\begin{Exercise}
Let $X,Y$ be two isometric spaces. Show that if $X$ is complete then $Y$ is also complete.
\end{Exercise}

\begin{Answer} 
Take an isometry $T:X\to Y$, and a Cauchy sequence $y_n$ in $Y$. Take $\varepsilon > 0$, and $N > 0$ such that for all $n,m > N$ we have that $\norm{y_n - y_m} < \varepsilon$. Then: 
\[  \norm{T^{-1}y_n - T^{-1} y_m} = \norm{T^{-1}\left(y_n - y_m\right)} = \norm{y_n -y_m} < \varepsilon \]
so $x_n = T^{-1}y_n$ is a Cauchy sequence, and since $X$ is complete, it has a limit $x \in X$. Now $\norm{y_n - Tx} = \norm{T\left(x_n - x\right)} \to 0$, so $y = Tx$ is a limit for $y_n$ in $Y$, so $Y$ is also complete. 
\end{Answer}

\begin{Exercise*}
Let $\left\{e_r\right\}_{r\in\Lambda}$ be an uncountable orthonormal system in a Hilbert space $H$. Prove Bessel's inequality and show that for every $v\in H$ we get $\inpr{v}{e_r} \neq 0$ for at most a countable subset of $\left\{e_r\right\}_{r\in\Lambda}$
\end{Exercise*}

\begin{Answer}
Take a finite subsuet $e_{r_1}, \dots, e_{r_n}$. Then we have that:
\begin{equation}
\begin{split}
0 & \leq \norm{x - \sum_{i=1}^{n}\inpr{x}{e_{r_i}}e_{r_i}}^2 \\
& = \inpr{x - \sum_{i=1}^{n}\inpr{x}{e_{r_i}}e_{r_i}}{x - \sum_{i=1}^{n}\inpr{x}{e_{r_i}}e_{r_i}} \\
& = \norm{x}^2 + \norm{\sum_{i=1}^{n}\inpr{x}{e_{r_i}}e_{r_i}}^2 -2\Re\inpr{x}{\sum_{i=1}^{n}\inpr{x}{e_{r_i}}e_{r_i}} \\
& = \norm{x}^2 + \sum_{i=1}^n\abs{\inpr{x}{e_{r_i}}}^2 -2\Re\sum_{i=1}^{n}\abs{\inpr{x}{e_{r_i}}}^2 \\
& = \norm{x}^2 - \sum_{i=1}^{n}\abs{\inpr{x}{e_{r_i}}}^2
\end{split}
\end{equation}

Now, consider $s = \sup \left\{\sum_{i=1}^{n}\abs{\inpr{x}{e_{r_i}}}^2:r_1,\dots,r_n \in \Lambda\right\} = \sup S$. We know that $s \leq \norm{x}^2$, so it's finite. Take a sequence $s_n \in S$ with $s_n \nearrow s$. Then we can take finite subsets $I_n \subset \Lambda$ with $s_n = \sum_{r\in I_n}\abs{\inpr{x}{e_r}}^2$. Consider $I = \bigcup_{n=1}^{\infty} I_n$. It's at most countable, as a countable union of finite sets. Suppose $\abs{\inpr{x}{e_{r^\prime}}} \neq 0$ for some $r^\prime \notin I$. Set $I_n^\prime = I_n \cup \left\{r^\prime\right\}$. So $s_n^\prime = \sum_{r\in I_n^\prime}\abs{\inpr{x}{e_r}}^2 = s_n + \abs{\inpr{x}{e_{r^\prime}}}^2$. So $s_n^\prime = s_n + \abs{\inpr{x}{e_{r^\prime}}}^2 \nearrow s + \abs{\inpr{x}{e_{r^\prime}}}^2 > s$, which is a contradiction, by the definition of $s$.

So the set of $r$ for which $\abs{\inpr{x}{e_{r^\prime}}}^2$ is at most a countable subset of $\Lambda$.
\end{Answer}

\begin{Exercise}
Let $H$ be a Hilbert space, $\left\{v_n\right\}_n$ a sequence of linearly independent vectors. Assume that $\sum_na_n v_n= 0$. Prove or Disprove: $a_n= 0$ for every $n$.
\end{Exercise}

\begin{Answer}
Consider the following example: $v_1 = \frac{1}{2}e_2$, $v_n = -\frac{1}{n}e_n + \frac{1}{n+1}e_{n+1}$ for $n>1$ in $\ell_2$. Then $\left\{v_n\right\}_n$ is a sequence on linearly independent vectors, but $\sum_{k=1}^{n}v_n = \frac{1}{n+1}e_{n+1} \to 0$ when $n\to \infty$.
\end{Answer}

\end{document}
